\documentclass[10pt]{article}

\begin{document}
\title{Project 1 Report}
\author{User~1 and User~2\\\footnotesize\texttt{user@uio.no}}
\date{}
\maketitle

\begin{abstract}
  This is an example. Use this template as a starting point for writing your report for project 1. The abstract is usually between 100-400 words, and should summarize the contents of your report. In short, state the problem and the methods you have used, and give your results and conclusion.
\end{abstract}

\section{Introduction}
Use the introduction to describe the problem, the methods you will apply to investigate or solve it, any hypotheses you have, and outline your findings.
How you organize the text is up to you, but keep in mind that you are in a sense telling the story of how your project went (without the gritty details). You want to capture the interest of the reader and then keep them interested enough to read your whole paper!
Why is the problem relevant? How are the results useful? What are the implications of your results?
Imagine that you are trying to ``sell'' your ``product'' (i.e. your project and results).
Feel free to use the problem description in the \texttt{project-1} document as a starting point.

\section{Related Work}
In this section, you contextualize your project by referring to other research that inspired, is similar to, or is otherwise related to your own.
We don't expect you to give a super detailed overview of all research about adaptation methods and ViTs that exists out there, but you can for example try to find some research on the performance of LoRA in different settings.
Put your references in the \texttt{report.bib} file. There should be an example there already \cite{Test}.

\section{Method}
This section describes your methodology in detail. In particular, you should explain LoRA as well as any technical details about ViTs that are relevant.
You can use the text in the \texttt{project-1} document as a starting point.

\section{Experiments and Results}
In this section you outline the experiments you have ran, and present the results.
In general, we encourage the use of tables, graphs, and figures.

\section{Discussion}
Here you discuss the results of your experiments. Did it match your initial hypotheses? Why/why not? What are the implications of the results?

\section{Conclusion}
State your conclusion to tie up the report. This is alse where you can mention the direction of future work based on the results of this one.

\bibliographystyle{IEEEtran}
\bibliography{report}

\end{document}